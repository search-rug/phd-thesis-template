\chapter{Introduction}
\label{c1}

\PARstart{T}{his}~chapter elaborates on the main concepts of this PhD project,
setting the context of the research work 
and describing the overall research design.

%
%
%

\section{Guidelines for the Introduction Chapter}
\label{sec:c1:guidelines}

You should shop around and look at theses that ask your supervisor for suggestions.
That said, here is an example of how you can structure your thesis.
You are likely to find that other theses from the group use similar structure.

\begin{itemize}
    \item Context
    \item Research Design
    \begin{itemize}
        \item Problem Statement
        \item Problem Decomposition
        \item Research Methods
    \end{itemize}
    \item Thesis Overview
\end{itemize}

\subsection{Context}
\label{sec:c1:context}

The \textbf{context} should allow a reader to understand enough
about the scientific field in order to be able to read the thesis.
It is not necessary to describe all of CS or all of SE. Go from general
(e.g. Software architecture) to specific (e.g. Technical Debt
in Software Architecture). You may find it easier to break the context down
into a sequence of sections.

\subsection{Research Design}
\label{sec:c1:rd}

The \textbf{problem statement} section should include
\begin{itemize}
    \item the causes of the problem;
    \item the problem itself, i.e. something that is wrong
    or can be improved; and
    \item the consequences of the problem
\end{itemize}

In the \textbf{problem decomposition}, a common problem
with by-publication theses is that the papers may not perfectly reflect
a coherent thread of research. It can be quite challenging to retro-fit
what was done to a consistent set of research questions. One solution
is to follow an established research framework like Wieringa's\footnote{
    Note that Wieringa's publications on Design Science tend to change things
    from year to year. A safe reference to consult is his book: R.J. Wieringa.
    Design Science Methodology for Information Systems
    and Software Engineering. Springer, 2014}
\emph{design science} framework for presenting the problem decomposition
in such thesis. Specifically, the problem decomposition
can be structured as follows:

\begin{itemize}
    \item Start from the problem statement (usually in the previous section)
    as the root of the decomposition tree.
    \item Describe sequentially a number of specific research questions
    that help to solve the problem statement. The research questions
    can be grouped if it helps to understand them better (e.g. 3 phases:
    understanding, exploring, contributing).
    \item Each specific research question should be mapped
    to exactly one chapter.
    \item Each specific research question should be linked to the next one.
    \begin{itemize}
        \item The first one is usually ``what does this concept mean?''
        \item The link to the next is found by asking: ``now that we know
        the answer to this question, this leads us to ask the following \dots''
    \end{itemize}
    \item Do not go into the details of the solution or the answer
    to the question. Only provide hints of the answer when linking questions
    to each other.
\end{itemize}

The \textbf{research methods} section provides an overview
of the empirical methods used in the thesis and a mapping of methods
to thesis chapters.

\subsection{Thesis Overview}
\label{sec:c1:overview}

The \textbf{thesis overview} section should
\begin{itemize}
    \item have the publications explicitly listed and mapped to thesis chapters;
    \item have the contributions explicitly listed; and
    \item explain what is your own contribution in papers with co-authors
    other than your supervisor(s) (if applicable), e.g. ``In this work,
    I took the lead in designing the proposed framework and the case study,
    conducting the case study in industry,
    performing data extraction and analysis, and writing the manuscript.''
    You do not need to explain the contribution of those other co-authors.
\end{itemize}

%
%
%

\section{\LaTeX{} Formatting}
\label{sec:c1:latex}

There are some great resources to help with \LaTeX,
so this guide will not include general instructions.
For those, consult the Latex Wikibook\footnote{\url{https://en.wikibooks.org/wiki/LaTeX}}
and the excellent guidelines by \cite{spinellis:latexadvice}.

That said, there are some small pointers that are particular to this template.
These are mentioned in the subsection below.

\subsection{References}
\label{sec:c1:References}

You may be used to write papers using numbered references (e.g. IEEE).
You are free to stick to such reference model in the thesis if you want.
However, this template uses \texttt{kluwer} (a style from the Harvard family),
which is a more common style for books (like your thesis).
If you want to use \texttt{kluwer}, you need to adjust your citations
according to the type of citation:

\vspace{0.5cm}
\noindent
\textbf{Indirect citation}

You are probably used to cite sources indirectly in a sentence.
For example, the code ``\verb|... in related work~\cite{source}|''
would generate something like ``\dots in related work~[2].''
To cite sources indirectly using styles like \texttt{kluwer},
you must use the command \verb|\citep{source}| (notice the `p').
The output will look as follows: \citep{spinellis:latexadvice},
i.e. \texttt{author, year} between parentheses.

\vspace{0.5cm}
\noindent
\textbf{Direct citation}

When you make a direct citation, you are probably used to write
the author manually and add an indirect citation. For example,
``\texttt{according to Author et} \verb|al.~\cite{source}|.''
To cite sources directly using styles like \texttt{kluwer},
you must use the command \verb|\cite{source}| (without `p')
and NOT add the author manually (this is done by the style).
For example, ``\verb|according to \cite{source}|''
will look as follows: ``according to \cite{spinellis:latexadvice}.''

\subsection{Tables}
\label{sec:c1:latex_tables}

Avoid cumbersome table formatting (e.g. using vertical lines).
In general, simpler is better.
\Cref{tab:c1:simpleexample} and \Cref{tab:c1:multicolexample} are some examples.
Also, look at the tables' source for code formatting tips.

\begin{table} [h]
    \caption{Example of a simple table (based on~\cite{spinellis:latexadvice})}
    \label{tab:c1:simpleexample}
    
    \noindent
    \centering
    %\footnotesize%if needed

    % put the header on a \parbox to center and break lines
    \newcommand{\hb}[2]{\parbox[c][0.8cm][c]{#1}{\centering #2}}
    
    \begin{tabular}{lccccc}
        \toprule
        % Column labels (using separate lines to align label with data)
        \hb{2cm}{Command\\and options}
                & \hb{2cm}{Input\\requirements}
                        & Matched
                                & Connected
                                        & Matched
                                                & Connected \\
        \midrule
        tr -cs  & 1     & \X    & --    & \V    & 1 \\
        sort w  & 0     &       & --    & \X    & 0 \\
        fmt     & 1     & \X    & --    & \V    & 1 \\
        tr A-Z  & 1     & \V    & 1     &       & 1 \\
        sort -u & fmt   & \X    & --    & \V    & 1 \\
        \bottomrule
    \end{tabular}
\end{table}


\begin{table} [h]
    \caption{Example of multi-column table}
    \label{tab:c1:multicolexample}
    
    \noindent
    \centering
    
    \begin{tabular}{lSSSS}
        \toprule
        {}                & \multicolumn{2}{c}{\textbf{Group 1}} 
                                                    & \multicolumn{2}{c}{\textbf{Group 2}} \\
                            \cmidrule(r){2-3}         \cmidrule(r){4-5}
        {}                & {Average}  & {Max}      & {Average}   & {Max} \\
        \midrule
        \textbf{Factor 1} &   853      &  2443      &   1760      &   3799      \\
        \textbf{Factor 2} & 14983      & 74658      & 110881      & 316552      \\
        \textbf{Factor 3} &    18      &    85      &     89      &    319      \\
        \textbf{Rate}     &    51.91\% &    86.34\% &     44.44\% &     81.08\% \\
        \bottomrule
    \end{tabular}
\end{table}

\subsection{Figures}
\label{sec:c1:latex_figures}

The main tips in a nutshell are as follows.
\begin{itemize}
    \item Put all your figures inside the \texttt{figs} folder,
          and use subfolders to organize them per chapter.
    \item Prefer vectorized images (e.g. PS or PDF).
          Bear in mind that saving a PNG or JPG as PDF
          will not vectorize your image.
\end{itemize}

See \Cref{fig:c1:simpleexample} for a simple example.
\Cref{fig:c1:allauthors} shows an example using sub-figures.
Notice that you can also cite them individually
(see \Cref{fig:c1:auth1}, \Cref{fig:c1:auth2} and \Cref{fig:c1:auth3}).

\begin{figure}
    \centering
    \includegraphics[width=.2\textwidth]{figs/cv/author}
    \caption{A figure example}
    \label{fig:c1:simpleexample}
\end{figure}

\begin{figure}
    \centering
    \begin{subfigure}[b]{0.2\textwidth}
        \centering
        \includegraphics[width=\textwidth]{figs/cv/author}
        \caption{Author 1}
        \label{fig:c1:auth1}
    \end{subfigure}
    \hfill
    \begin{subfigure}[b]{0.2\textwidth}
        \centering
        \includegraphics[width=\textwidth]{figs/cv/author}
        \caption{Author 2}
        \label{fig:c1:auth2}
    \end{subfigure}
    \hfill
    \begin{subfigure}[b]{0.2\textwidth}
        \centering
        \includegraphics[width=\textwidth]{figs/cv/author}
        \caption{Author 3}
        \label{fig:c1:auth3}
    \end{subfigure}
       \caption{All authors}
       \label{fig:c1:allauthors}
\end{figure}
